\phantomsection\section*{ВВЕДЕНИЕ}\addcontentsline{toc}{section}{ВВЕДЕНИЕ}

% \noindent
На сегодняшний день компьютерная графика является неотъемлемой частью
нашей жизни и используется повсеместно. Ей находится применение в самых разных
областях человеческой деятельности: она используется в науке, в бизнесе, в
кино, играх и везде, где нужно визуальное представление информации на
электронном дисплее. \cite{kurov}

% \noindent
Часто перед разработчиками графического программного обеспечения
стоит задача синтеза реалистического изображения. Для решения этой задачи
существует множество алгоритмов, но, как правило, алгоритмы, дающие наилучшие
результаты являются наиболее трудозатратными как по объёму требуемой памяти,
так и по количеству требуемых вычислений, потому что учитывают множество
световых явлений (дифракция, интерференция, преломление, поглощение,
множественное отражение). Поэтому, в зависимости от задачи и от имеющихся
вычислительных мощностей, программистам приходится выбирать наиболее
целесообразные в их случае алгоритмы и жертвовать либо временем генерации
кадра, либо его реалистичностью.

% \noindent
Но генерация реалистического изображения -- это лишь одна из многих
задач, которые стоят перед программистами графических приложений. Часто
требуется, чтобы разрабатываемое приложение было интерактивным -- давало
возможность как-то взаимодействовать с объектами, находящимися в виртуальном
пространстве: перемещать их, сталкивать друг с другом, разрушать,
деформировать.

% \noindent
Моей целью во время практики будет изучение и выбор алгоритмов для
разработки программы моделирования упругих столкновений объектов. Исследуемые
алгоритмы можно будет разделить на две категории: графические -- относящиеся к
изображению объектов на экране (удаление невидимых ребер и поверхностей,
освещение) и физические (обнаружение столкновений объектов, реагирование на
столкновение и оптимизация этих процессов). % XXX

% \noindent
% Для достижения поставленной цели требуется решить следующие задачи:

% \begin{itemize}
%     \item TODO
% \end{itemize}
