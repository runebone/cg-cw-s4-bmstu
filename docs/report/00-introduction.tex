\phantomsection\section*{ВВЕДЕНИЕ}\addcontentsline{toc}{section}{ВВЕДЕНИЕ}

На сегодняшний день компьютерная графика является неотъемлемой частью нашей жизни и используется повсеместно.
Ей находится применение в самых разных областях человеческой деятельности: она используется в науке, в бизнесе, в кино, играх и везде, где нужно визуальное представление информации на электронном дисплее.

Перед разработчиками графического программного обеспечения часто стоит задача синтеза реалистического изображения.
Для решения этой задачи существует множество алгоритмов, но, как правило, алгоритмы, дающие наилучшие результаты являются наиболее трудозатратными как по объёму требуемой памяти, так и по количеству требуемых вычислений, потому что учитывают множество световых явлений (дифракция, интерференция, преломление, поглощение, множественное отражение).
Поэтому, в зависимости от задачи и от имеющихся вычислительных мощностей, программистам приходится выбирать наиболее целесообразные в их случае алгоритмы и жертвовать либо временем генерации кадра, либо его реалистичностью.

Но генерация реалистического изображения --- это лишь одна из многих задач, которые стоят перед программистами графических приложений.
При разработке игр или приложений для моделирования физики твёрдого тела возникает задача обнаружения коллизий (столкновений) между объектами виртуального пространства и реагирования на них, например, разрушение объектов в результате столкновения, деформация объектов, или их упругое соударение.

В данной курсовой работе было решено создать программу --- песочницу, в которой пользователь сможет размещать объекты в виртуальном пространстве, изменять их свойства (такие как размер, местоположение, цвет, масса), задавать им начальные скорости; после чего наблюдать их перемещение и столкновения.

Цель работы --- разработка программы для моделирования упругих столкновений объектов в пространстве.

Для достижения поставленной цели требуется решить следующие задачи:
\begin{itemize}
    \item описать свойства объекта, которыми он должен обладать для моделирования его движения и столкновения с другими объектами;
    \item проанализировать существующие способы представления объектов и обосновать выбор наиболее подходящего для решения поставленной задачи;
    % \item проанализировать существующие алгоритмы построения изображения и обосновать выбор тех из них, которые в наибольшей степени подходят для решения поставленной задачи;
    \item проанализировать существующие алгоритмы удаления невидимых линий и поверхностей и обосновать выбор тех из них, которые в наибольшей степени подходят для решения поставленной задачи;
    \item проанализировать существующие алгоритмы обнаружения коллизий и обосновать выбор тех из них, которые в наибольшей степени подходят для решения поставленной задачи;
    % \item проанализировать существующие алгоритмы разрешения коллизий и обосновать выбор тех из них, которые в наибольшей степени подходят для решения поставленной задачи;
    \item проанализировать существующие модели освещения и обосновать выбор модели, наиболее подходящей для решения поставленной задачи;
    \item реализовать выбранные алгоритмы;
    \item разработать программное обеспечение для решения поставленной задачи;
    \item провести анализ производительности работы программы в зависимости от количества объектов на сцене и их типов.
\end{itemize}
