\section{Конструкторская часть}

% Кратко описать, что будет в конструкторской части

В данной части будут приведены требования к программному обеспечению, на формальном языке будут описаны алгоритмы, которые будут реализованы при разработке программного обеспечения, а также будет обоснован выбор типов и структур, которые будут использованы при разработке, и приведена общая архитектура разрабатываемой программы.

\subsection{Требования к программному обеспечению}

% Составить ТЗ, что должна делать программа

Разрабатываемое программное обеспечение должно предоставлять пользователю следующую функциональность:
\begin{itemize}
    \item добавление объекта на сцену (куб, сфера, чайник);
    \item выбор объекта сцены с помощью клавиатуры;
    \item изменение цвета выбранного объекта;
    \item изменение геометрических свойств выбранного объекта (положение в пространстве, поворот, увеличение);
    \item изменение физических свойств выбранного объекта (масса, скорость, ускорение, сила);
    \item изменение значения гравитации;
    \item перемещение и поворот камеры с помощью клавиатуры и мыши.
\end{itemize}

При этом разрабатываемая программа должна удовлетворять следующим требованиям:
\begin{itemize}
    \item программа должна генерировать кадр не менее, чем за $\frac{1}{60}$ секунды;
    \item никакие действия пользователя не должны приводить к аварийному завершению программы.
\end{itemize}

\subsection{Разработка алгоритмов}

Далее на формальном языке будут описаны алгоритмы, которые будут реализованы при разработке проргаммного обеспечения.

\subsubsection{Общий алгоритм работы программы}

% \subsubsection{Алгоритм, использующий z-буфер}

\subsubsection{Алгоритм AABB}

\subsubsection{Алгоритм GJK}

\subsubsection{Алгоритм EPA}

\subsubsection{Модель освещения Фонга}

\subsection{Выбор типов и структур данных}

\subsection{Общая архитектура разрабатываемой программы}

% Классы

\subsection*{Вывод}

% Всё, что надо планировалось, было сделано, схемы алгоритмов разработаны и т.д.
