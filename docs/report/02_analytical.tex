% После прочтения аналит. части человек должен сделать вывод о применимости конкретного алгоритма в его конкретной ситуации

% В аналитической части описать условия применимости алгоритма в принципе (примеры и контрпримеры - при наличии доступного объема текста)

% Таблица в конце аналитического раздела

% Констр. + аналит.: на вход - информация о полигонах

\newpage
\section{Аналитический раздел}

\subsection{Представление объектов}

Существует множество способов представления трехмерных объектов в пространстве.
Наиболее популярные из них: каркасное, поверхностное и твердотельное.

\textbf{Каркасная модель} -- объекты представляются в виде набора вершин и
ребер (точек и линий). Самый простой способ и наименее требовательный к памяти
из всех рассматриваемых.

\textbf{Поверхностная модель} -- объекты представляются в виде набора
поверхностей. Поверхность, в свою очередь, также можно задать несколькими
способами, например, уравнением или набором плоскостей, аппроксимирующим
исходную поверхность. Сложные объекты часто бывает сложно или даже невозможно
описать в виде системы уравнений, а аппроксимировать, с другой стороны,
возможно всегда.

\textbf{Твердотельная модель} -- % TODO

% \noindent
% \begin{adjustbox}{width=1\textwidth}
%     \begin{tabular}{|p{.25\textwidth}|p{.25\textwidth}|p{.25\textwidth}|p{.25\textwidth}|}
%         \hline
%         Модель
%         &
%         Реалистичность
%         &
%         Память
%         &
%         Простота использования
%         \\

%         \hline
%         Каркасная
%         &
%         --
%         &
%         Хранится только информация о вершинах и ребрах.
%         &
%         Просто
%         \\

%         \hline
%         Поверхностная
%         &
%         +
%         &
%         TODO
%         &
%         Просто
%         \\

%         \hline
%         Твердотельная
%         &
%         +
%         &
%         TODO
%         &
%         TODO
%         \\
%         \hline
%     \end{tabular}
% \end{adjustbox}

TODO выбор модели

\subsection{Формализация объектов синтезируемой сцены}

\begin{itemize}
    \item \textbf{Точечный источник света} -- точка в пространстве, излучающая свет во всех направлениях. Интенсивность света убывает по мере удаления от источника. Точечный источник света характеризуется:
        \begin{enumerate}[label=(\alph*)]
            \item Положением в пространстве (трехмерные координаты)
            \item Цветом ({RGB})
            \item Интенсивностью \textbf{I} (действительное число от $0$ до $1$)
            % \item Начальной интенсивностью \textbf{I} (действительное число от $0$ до $1$)
            % \item Функцией спада интенсивности в зависимости от расстояния до источника света (постоянная, линейная, квадратичная)
            % \item Радиусом действия \textbf{R} (на границе радиуса действия функция спада интенсивности обращается в нуль)
            \item Направлением (когда источник света расположен в бесконечности)
        \end{enumerate}
        % Функции спада интенсивности имеют вид:
        % \begin{itemize}
        %     \item Постоянная: $ f(x) = {I},\ 0 \leq x \leq {R} $
        %     \item Линейная: $ f(x) = {I} \cdot \left( 1 - \frac{x}{{R}} \right),\ 0 \leq x \leq {R} $
        %     \item Квадратичная: $ f(x) = {I} \cdot \left( 1 - \frac{x}{{R}} \right)^2,\ 0 \leq x \leq {R} $
        % \end{itemize}
    \item \textbf{Объект сцены} -- набор трехмерных примитивов, формирующих полигональную сетку. Объект сцены характеризуется:
        \begin{enumerate}[label=(\alph*)]
            \item Положением в пространстве (трехмерные координаты)
            \item Ориентацией в пространстве (матрица модели)
            \item Цветом ({RGB})
        \end{enumerate}
    \item \textbf{Земля} -- {объект сцены}, представляющий собой рельефную территорию.
    \item \textbf{Плоскость Земли} -- горизонтальная плоскость (параллельная {XY}), проходящая через самую низкую точку {Земли}.
    \item \textbf{Камера} характеризуется:
        \begin{enumerate}[label=(\alph*)]
            \item Положением в пространстве (трехмерные координаты)
            \item Направлением взгляда (трехмерный вектор)
            \item Ориентацией в пространстве (трехмерный вектор, указывающий, в каком направлении у камеры верх)
            \item Расстояниями до ближней и дальней граней пирамиды видимости
            \item Углом обзора
            \item Соотношением сторон
        \end{enumerate}
\end{itemize}

\subsection{Анализ алгоритмов удаления невидимых линий и поверхностей}
% \subsection*{Вывод} (таблица)

Алгоритмы удаления невидимых линий и поверхностей можно разделить на две
группы. Одни работают в объектном пространстве -- для каждого объекта
проверяется, заслоняют ли его другие объекты или нет. Таким образом,
потребуется $O(n^2)$ сравнений объектов. К таким алгоритмам относится,
например, алгоритм Робертса. Другие работают в пространстве изображения -- для
каждого пикселя изображения определяется, какой объект сцены в нём виден. То
есть, потребуется $O(Nn)$ сравнений, где $N$ -- количество пикселей. Таковы
алгоритмы трассировки лучей, Варнока или Z-буфера.

На первый взгляд может показаться, что, если мы редко имеем дело с более чем
$1920 \times 1080$ объектами, то эффективнее будет использовать алгоритм,
работающий в объектном пространстве, чтобы было меньше сравнений, однако это
далеко не всегда так. TODO

\subsubsection{Алгоритм Робертса}

Одно из требований алгоритма -- тела должны быть выпуклыми. Поэтому, если
какие-то из объектов сцены не выпуклые, то их следует разбить на выпуклые
составляющие. На последующих этапах алгоритма удаляются:
\begin{enumerate}
    \item Невидимые рёбра, экранируемые самим телом;
    \item Невидимые рёбра, экранируемые другими телами сцены;
    \item Новые невидимые рёбра, возникающие при <<протыкании>> тел друг
        другом.
\end{enumerate}

К преимуществам данного алгоритма можно отнести его математическую точность и
простоту идеи.

К недостаткам -- следующие:
\begin{enumerate}
    \item Трудности при работе с невыпуклыми объектами; их нужно разделять на
        выпуклые составляющие, что само по себе задача трудоёмкая.
    \item Для каждого выпуклого тела составляется матрица размерности $4 \times
        N$, где $N$ -- количество граней тела. Если в программе используются
        модели с большим количеством полигонов, эта матрица может быть очень
        большой, и выполнение операций над ней может значительно повлиять на
        время генерации кадра.
    \item После разбиения объектов на выпуклые составляющие, количество
        объектов, для которых нужно выполнить проверку перекрытия возрастает,
        что увеличивает время выполнения алгоритма.
\end{enumerate}


\subsection{Анализ алгоритмов обнаружения коллизий}
