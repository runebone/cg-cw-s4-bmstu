\section{Технологическая часть}

В данной части будет обоснован выбор графического API, языка программирования и среды разработки, которые будут использоваться при разработке программного обеспечения.
Также будет приведена UML-диаграмма классов, описывающая структуру программы.
Будет продемонстрирован интерфейс, и будут приведены примеры работы программы.

% @Cite LINKS!!!!!

\subsection{Средства реализации}

Далее будет обоснован выбор языка программирования и средств разработки, использованных при разработке программы.

\subsubsection{Выбор графического API}

В качестве используемого графического API был выбран OpenGL по следующим причинам:
\begin{itemize}
    \item кроссплатформенность (в отличие от DirectX);
    \item простота инициализации (в отличие от Vulkan);
    \item использование вычислительных мощностей графического ускорителя (в отличие от рендеринга на процессоре);
    \item спецификация OpenGL реализована в драйверах большинства массовых видеокарт (NVIDIA, AMD, Intel), в связи с чем приложение возможно будет запустить практически на любом персональном компьютере.
\end{itemize}

\subsubsection{Выбор языка программирования}

В качестве используемого языка программирования был выбран C++ по следующим причинам:
\begin{itemize}
    \item язык широко используется при разработке графических приложений;
    \item наличие комплияторов, генерирующих высокопроизводительный исполняемый код;
    \item язык типизирован, в связи с чем в процессе разработки возникает меньше ошибок времени выполнения;
    \item наличие библиотек для обеспечения доступа к функциям OpenGL (glad), а также для создания окон и управления вводом (GLFW);
    \item наличие математических библиотек (glm);
    \item язык поддерживается отладчиками gdb и RenderDoc;
    \item наличие кроссплатформенной утилиты для автоматической сборки программы cmake.
\end{itemize}

\subsubsection{Выбор среды разработки}

В качестве среды разработки был выбран текстовый редактор Neovim по следующим причинам:
\begin{itemize}
    \item высокая отзывчивость в отличие от графических сред разработки, таких как Visual Studio, Visual Studio Code, Clion, QtCreator;
    \item полная поддержка vim-движений, что ускорит навигацию по исходному коду проекта;
    \item возможность использования протокола Language Server, что позволит определять наличие ошибок времени компиляции в коде без необходимости компиляции проекта;
    \item наличие расширений, ещё больше ускоряющих процесс разработки и навигации по исходному коду проекта (harpoon, vim-fugitive, nvim-tree, nvim-cmp, nvim-treesitter, nvim-lspconfig, telescope-nvim, luasnip, vim-surround, vim-commentary, friendly-snippets).
\end{itemize}

\subsection{Структура программы}

\subsection{Интерфейс}

\subsection{Работа программы}

Далее будут приведены замечания относительно работы программы.

% Как программа работает, что учитывается, что нет

\subsection{Демонстрация работы программы}
