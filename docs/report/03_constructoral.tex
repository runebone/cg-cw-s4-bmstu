% После прочтения конструкторской части человек должен иметь возможность закодировать алгоритм на ЯП

\newpage
\section{Конструкторский раздел}

\subsection{Требования к программному обеспечению}

Пользователь должен иметь следующие возможности:
\begin{itemize}
    \item Создание объектов: куб, шар, чайник, прямоугольный параллелепипед.
    \item Изменение свойств объектов: размер, местоположение, цвет, масса, начальная скорость.
    \item Удобное управление камерой и возможность рассмотра сцены с разных сторон.
    \item Изменение значения гравитации.
    \item Запуск режима моделирования, в котором программа будет моделировать движение и столкновение объектов друг с другом.
\end{itemize}

Требования к программе:
\begin{itemize}
    \item Программа должна генерировать кадр не более, чем за $\frac{1}{24}$ секунды.
    \item Программа должна учитывать освещение сцены при моделировании движения объектов.
    \item Никакие действия пользователя не должны приводить к аварийному завершению программы.
\end{itemize}

% \subsection{Общий алгоритм работы программы}% (построения изображения)}

% \subsection{Алгоритм TODO}

% \subsection{Выбор используемых типов и структур данных}
