\newpage
\section{Технологический раздел}

\subsection{Выбор и обоснование языка программирования и среды разработки}

В качестве языка программирования был выбран \textbf{C++} по следующим причинам:
\begin{itemize}
    \item Имеется опыт разработки на данном языке
    \item \textbf{C++} -- один из основных языков программирования, используемых для создания графических программ; огромное количество графического программного обеспечения в мире написано именно на языке \textbf{C++}
    \item Компилятор языка переводит текст программы в машинный код, который исполняется быстро
    \item Язык поддерживает объектно-ориентированную парадигму программирования, что позволит создавать удобные абстракции во время написания кода
    \item Существуют библиотеки (\textbf{glad}, \textbf{GLEW}), предоставляющие доступ к функциям \textbf{OpenGL}
    \item Наличие математических библиотек, использование которых ускорит процесс разработки (\textbf{glm}, \textbf{eigen})
    \item Язык поддерживается отладчиком \textbf{gdb}
    \item Имеется возможность профилирования программы с помощью утилиты \textbf{gprof}
    \item Наличие графического отладчика \textbf{RenderDoc}
    \item Наличие кроссплатформенной утилиты для автоматической сборки программы \textbf{cmake}
\end{itemize}

В качестве среды разработки был выбран \textbf{neovim} по следующим причинам:
\begin{itemize}
    \item Поддержка комбинаций клавиш текстового редактора \textbf{vim}, что
        позволит быстрее писать и редактировать написанный код
    \item Поддержка протокола \textbf{Language Server}, что позволит определять
        наличие ошибок времени компиляции в коде непосредственно во время
        написания кода, что ускорит процесс разработки
    \item Наличие множества плагинов ещё больше ускоряющих процесс разработки:
        \textbf{vim-fugitive} для использования системы контроля версий
        \textbf{git} не покидая среду разработки, \textbf{harpoon} и
        \textbf{telescope-nvim} для быстрой навигации между файлами проекта,
        \textbf{nvim-treesitter} для подсветки синтаксиса,
        \textbf{nvim-lspconfig} для использования протокола \textbf{Language
        Server}, \textbf{nvim-cmp}, \textbf{luasnip},
        \textbf{friendly-snippets} и \textbf{cmp-luasnip} для автодополнения
        кода
\end{itemize}

% TODO: дебаггинг и профилирование
