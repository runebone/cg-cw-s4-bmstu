\phantomsection\section*{ЗАКЛЮЧЕНИЕ}\addcontentsline{toc}{section}{ЗАКЛЮЧЕНИЕ}

В ходе выполнения данного курсового проекта, были
\begin{itemize}
    \item описаны свойства объекта, которыми он должен обладать для моделирования его движения и столкновения с другими объектами;
    \item проанализированы существующие способы представления объектов и был обоснован выбор наиболее подходящего для решения поставленной задачи;
    \item проанализированы существующие алгоритмы обнаружения коллизий и был обоснован выбор тех из них, которые в наибольшей степени подходят для решения поставленной задачи;
    \item проанализированы существующие модели освещения и был обоснован выбор модели, наиболее подходящей для решения поставленной задачи;
    \item реализованы выбранные алгоритмы.
\end{itemize}

Также было разработано программное обеспечение для решения поставленной задачи, и был проведён анализ зависимости времени генерации кадра от: количества треугольников, их которых состоят модели объектов сцены; количества столкновений объектов сцены; количества вызовов функций графического ускорителя.

В результате исследования выяснилось, что в наибольшей степени на время генерации кадра влияет сильное (в 6077 -- 6270 раз) увеличение количества вызовов функций графического ускорителя.
При увеличении количества вызовов OpenGL-функций отрисовки в 6077 -- 6270 раз, время генерации кадра увеличивается в среднем в 26 раз.

Все задачи для достижения цели были решены, и цель работы была достигнута.
